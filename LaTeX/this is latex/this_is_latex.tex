\documentclass{article}
\usepackage[utf8]{inputenc} %support encoding for various symbols
\usepackage[T1]{fontenc} %support fonts
\usepackage[french]{babel} %set language [funky, might have to restart VScode if changed]
\usepackage{lmodern} %font typeface
\usepackage[nottoc]{tocbibind} %index document elements
%%% Use to put unnumbered sections in the TOC
\newcommand*{\nsection}[1]{
    \section*{#1}
    \addcontentsline{toc}{section}{#1}
}
\newcommand*{\nsubsection}[1]{
    \subsection*{#1}
    \addcontentsline{toc}{subsection}{#1}
}

\usepackage{multicol}
\usepackage{lipsum} %filler "lorem ipsum" text
\newcommand{\lip}[1]{\lipsum[#1][1-3]}
\usepackage{xcolor} %colours
\definecolor{dark green}{HTML}{007300}
\definecolor{enby}{HTML}{9b59d0}
\usepackage{tcolorbox} %text blocks [very powerful]
\tcbuselibrary{most, theorems}
\tcbset{enhanced jigsaw} %define global tcb settings

\newcommand{\lrmargin}{1in} %define left-right margin value
\usepackage{geometry} %page layout
\geometry{
    a4paper,
    top=2cm,
    bottom=1in,
    left=\lrmargin,
    right=\lrmargin,
}
\usepackage{titling} %better control in titles
\usepackage[parfill]{parskip} %paragraph spacing
\usepackage{setspace} %line spacing
\usepackage{ragged2e} %better text alignement
\usepackage{epigraph} %for quotations
\usepackage{fancyhdr, lastpage} %for headers and footers, adds the last page number as a ref
\pagestyle{fancy} %apply style
\fancyhead{} %remove defaut headers
\fancyfoot{} %remove defaut footers
\fancyhead[L]{This is \LaTeX}
\fancyhead[R]{Astro}
\fancyfoot[C]{\thepage\ sur\ \pageref*{LastPage}}
\usepackage{fancyvrb} %verbatim code
\usepackage{listings} %code blocks
\usepackage{enumitem} %customises lists
\setlist[itemize,1]{label=$\bullet$}
\setlist[itemize,2]{label=-}
\setlist[itemize,3]{label=$\rightleftharpoons$}

\usepackage{wrapfig} %wrap text around figures and tables
\usepackage{array} %better tables
\usepackage{booktabs} %better table lines
\newcommand{\trule}{\addlinespace[1pt]\toprule}
\usepackage{graphicx, float} %required for inserting images
\graphicspath{{./images/}} %specifies filepath for images
\newcommand{\anarchy}{\includegraphics[width=12pt, decodearray={0 0 0 0 0 0}]{anarchy_symbol.png}}
\usepackage{tikz,tikz-cd} %graphs, commutative diagrams
%\usepackage[compat=1.1.0]{tikz-feynman} %feynman diagrams [%needs luatex%]
\newcommand*{\pretable}[4]{
    % 1=tablename; 2=column name; 3=error column name; 4=new column name
    \pgfplotstablecreatecol[
        create col/assign/.code={%
            \edef\entry{$\thisrow{#2} \pm \thisrow{#3}$}%
            \pgfkeyslet{/pgfplots/table/create col/next content}\entry
        }
    ]{#4}#1
}

\usepackage{amsmath, amssymb, amsthm, amsfonts, bm} %math stuff  
\usepackage[fixamsmath]{mathtools}  % Extension to amsmath
\newcommand{\refmath}[1]{\texorpdfstring{$\bm{#1}$}{#1}}
\newcommand{\refmathx}[2]{\texorpdfstring{$\bm{#1}$}{#2}}
\usepackage{physics} %physics notation
\newcommand{\si}[1]{\ \text{#1}}
\newcommand{\N}{\mathbb{N}}
\newcommand{\Z}{\mathbb{Z}}
\newcommand{\Q}{\mathbb{Q}}
\newcommand{\R}{\mathbb{R}}
\newcommand{\C}{\mathbb{C}}
\newcommand{\K}{\mathbb{K}}
\newcommand{\M}[3]{\mathbb{#1}^{#2\cp#3}}
% Symbol and utility packages
\usepackage{cancel}
\usepackage[nointegrals]{wasysym}

\theoremstyle{definition}
\newtheorem{dfn}{Définition}[section]
\theoremstyle{plain} %default
\newtheorem{lem}[dfn]{Lemme}
\newtheorem{thm}[dfn]{Théorème}
\newtheorem{prop}[dfn]{Proposition}
\newtheorem{cor}[dfn]{Corollaire}
\theoremstyle{remark}
\newtheorem{rem}[dfn]{Remarque}
%other environment [TBD]
\theoremstyle{definition}
\newtheorem{exe}[dfn]{Exemple}
\newtheorem{exo}[dfn]{Exercice}

\foreach\var in {dfn, lem, thm, prop, cor}{
    \tcolorboxenvironment{\var}{
    blanker, breakable, left=5mm, borderline west={1mm}{0pt}{black}
}}

% \newtcbtheorem
%     [number within=section, use counter from=dfn]% init options
%     {example}% name
%     {Example}% title
%     {%
%         colback = orange!5,
%         colframe = green!50!black!50!white,
%         fonttitle = \bfseries,
%     }% options
%     {ex}% prefix

\usepackage{footnotebackref}
\usepackage{hyperref} %hyperlink stuff [!!ALWAYS LAST!!]

\hypersetup{
    colorlinks = true,
    linkcolor = dark green,
    urlcolor = blue,
    pdftitle={This is LaTeX},
}
\urlstyle{same} %changes url font from monospace to normal

\usepackage[french]{cleveref} %after hyperref

\VerbatimFootnotes %enables \verb commands inside footnotes

\title{This is \LaTeX \\ \vspace{1pt}\large{A playground}}
\author{Astro}
\date{November 2023}

%this is a comment
\begin{document}

\maketitle
\tableofcontents

\section{Introduction}

\begin{wrapfigure}{r}[15pt]{0.25\textwidth}
	\includegraphics[width=0.9\linewidth]{picrew_catenby.png}
\end{wrapfigure}
\lipsum[1-2]


Hello world! I am groot. The flag illustrated in figure (\ref{fig:relanark}) is pretty \anarchy . $e^{i\pi} + 1 = 0$ where $e$ is:
\begin{equation} \label{eq:e_def}
	e = \lim_{n \to \infty} \qty(1 + \frac{1}{n})^n = \lim_{n \to \infty} \frac{n}{\sqrt[n]{n!}} = \sum_{n=0}^{\infty} \frac{1}{n!}
\end{equation}

\begin{tcolorbox}[breakable, colback=orange!5, colframe=enby!65!white!80!black, title=This is a tcolorbox]
	\lipsum[1]
\end{tcolorbox}

\begin{figure}[H]
	\centering
	\includegraphics[width=0.9\textwidth]{relationship_anarchy.png}
	\caption{relationship anarchy}
	\label{fig:relanark}
\end{figure}

\begin{prop}
	$\forall x,y \in \R^n, \quad \big|\|x\|-\|y\|\big|\le\|x-y\|$.
\end{prop}
\begin{proof}
	Ecrivons $x=x-y+y$, alors par l'inégalité triangulaire on a
	\[\|x\|=\|x-y+y\|\le\|x-y\|+\|y\| \iff \|x\|-\|y\|\le\|x-y\| \text{.}\]
	Réciproquement,	\[\|y\|=\|y-x+x\|\le\|y-x\|+\|x\| \iff \|y\|-\|x\|\le\|y-x\| \iff \|x\|-\|y\|\ge-\|x-y\| \text{.}\]
	On a donc\, $-\|x-y\|\le\|x\|-\|y\|\le\|x-y\|$. Ce qui montre que $\big|\|x\|-\|y\|\big|\le\|x-y\|$.
\end{proof}

\begin{enumerate}
	\item equation \labelcref{eq:e_def} is interesting
	      \begin{enumerate}
		      \item bla
		      \item bla
	      \end{enumerate}
	\item bla
\end{enumerate}

\section{Chapitre 1}

\begin{thm}[Chapter 1]
	this is chapter 1.\\ \lip{2}
\end{thm}
\begin{proof}
	the proof is left as an exercise to the reader
\end{proof}
\begin{cor}
	\lip{2}
	\begin{equation}
		\label{eq:multint}
		\iiint_a^b f(x,y,z) \, dx\,dy\,dz
	\end{equation}
\end{cor}

\begin{itemize}
	\item this is \textbf{bold text}
	\begin{itemize}
		\item bla
		\begin{itemize}
			\item bla 
		\end{itemize} 
	\end{itemize}
	\item this is \textit{italic text}
	\item this is \underline{underlined text}
	\item L'\cref{eq:e_def,eq:multint} sont intéressantes
\end{itemize}

\[x \notin \R \; \text{or} \; \M{R}{2}{3}\] %\N, \Z, \C, \K

\begin{table}[H]
	\centering
	\begin{tabular}{rcc}\toprule
		\textbf{header} & \textbf{column 1 (m/s)} & \textbf{column 2 (s)} \\\midrule
		cell 0          & cell 1                         & cell 2         \\
		cell 0          & cell 1                         & cell 2         \\
		cell 0          & cell 1                         & cell 2         \\
		cell 0          & cell 1                         & cell 2         \\
		\bottomrule
	\end{tabular}
	\caption{this is a table}
\end{table}



\section{Including \TeX \ files}
When to use \verb|\input| or \verb|\include|?
\vspace{-5pt}
\subsection{Input}
The \verb|\input{<filename>}| macro is basically the same as pasting the target code where the command was used.\\
Mentionable properties of \verb|\input| are:
\begin{itemize}
	\item You can use \verb|\input| basically everywhere with any content.
	It is usable in the preamble, inside packages and in the document.
	\item You can nest \verb|\input| macros.
	You can use \verb|\input| inside a file which is read using \verb|\input|.
	\item The only thing \verb|\input| does is to input the file.
	You don't have to worry about any side effects, but don't get any extra features.
\end{itemize}

\subsection{Include}
\verb|\include| does basically the following thing:
\begin{itemize}
	\item It uses \verb|\clearpage|\footnotemark \ before and after the content of the file. This ensure that its content starts on a new page of its own and is not placed together with earlier or later text.
	\item It opens a new .aux file for the given file.
	There will be a filename.aux file which contains all counter values, like page and chapter numbers etc\ldots (so they can be compiled separately). Such \textit{part} .aux files are read by the main .aux file.
	\item It then uses \verb|\input| internally to read the file's content.
\end{itemize}
\footnotetext{\verb|\clearpage| is just like \verb|\newpage| but it forces float objects to print before the new section.}
\vspace{5pt}
Mentionable properties of \verb|\include| are:

\begin{itemize}
	\item It can't be used anywhere except in the document and only where a page break is allowed.
	Because of the \verb|\clearpage| and the own .aux file \verb|\include| doesn't work in the preamble, or inside packages. Using it in restricted modes or math mode won't work properly, while \verb|\input| is fine there.
	\item You can't nest \verb|\include| files.
	You can't use \verb|\include| inside a file which is read by \verb|\include|.
	\item \textbf{Biggest benefit:} You can use \verb|\includeonly{<filename1>,<filename2>,...}| in the preamble to only include specific \verb|\include| files.
	Because the state of the document (i.e. above mentioned counter values) was stored in an own .aux file all page and sectioning numbers will still be correct. This is very useful in the writing process of a large document because it allows you to only compile the chapter you currently write on while skipping the others.
	There is also the excludeonly package which provides an \verb|\excludeonly| to exclude only certain files instead of including all other files.
\end{itemize}

\appendix %start appendix chapters
\section{Appendix}


\end{document}